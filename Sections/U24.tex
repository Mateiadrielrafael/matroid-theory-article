We will consider the uniform matroid $U_{2,4}$ and show that it is $\mathbb{F}_3$-representable, while it is not $\mathbb{F}_2$-representable. First, $U_{2,4}= (\{1,2,3,4\}, \mathcal{I})$ where the bases are all of the two element subsets of $\{1,2,3,4\}$, so the set of bases 

$$\mathcal{B} = \{\{1,2\}, \{1,3\}, \{1,4\}, \{2,3\}, \{2,4\}, \{3,4\}\}$$.

Suppose $U_{2,4}$ were $\mathbb{F}_2$-representable. Then there is some $A = \mat_{m \times 4}(\mathbb{F}_2)$, 

$$A = \begin{pmatrix}
    v_1 & v_2 & v_3 & v_4
\end{pmatrix}$$

so that $U_{2,4} \sim M[A]$. We have the following observation.

\begin{lemma}
\label{f2lema}
    Suppose a set of vectors $A = \{w_1, w_2, \cdots, w_r\} \subset \mathbb{F}_2^n$ is minimaly linarly dependent, that means that it is linarly dependent but any proper subset of it is linearly independent. Then $w_1 + w_2 + \cdots + w_r = 0$.
\end{lemma}

\begin{proof}
    Because $A$ is linearly dependent, there exists a set of scalars $\{a_1, a_2, \cdots, a_r\}\in \mathbb{F}_2$ not all zero such that
    
    $$\sum_{i=1}^r a_iv_i = 0.$$
    
    If for some $1\leq j \leq r$ we have $a_j = 0$ then we also have 
    
    $$\sum\limits_{\substack{i = 1 \\ i \neq j}} ^r a_iv_i = 0$$

    but not all out of $a_1, \cdots a_{j-1}, a_{j+1}, \cdots a_r$ are 0. This means that ${v_1, \cdots, v_{j-1}, v_{j+1}, \cdots, v_r}$ is linearly dependent, which is not true by assumption. Therefore for all $1\leq j \leq r$ we have $a_j \neq 0$. But since the field is $\mathbb{F}_2$ this forces $a_j = 1$ for all $j$. Hence we obtained what we want, namely

     $$\sum_{i=1}^r a_iv_i = 0 =  \sum_{i=1}^r 1 \cdot v_i = 0.$$
    
\end{proof}

If the matroid $U_{2,4} \sim M[A]$ for the above $A$ then we would have all of the three element subsets of vectors to be minimaly linearly dependent. This means by \ref{f2lema} that $v_1 + v_2 + v_3 = 0$ and $v_1 + v_2 + v_3 = 0$. However, we are in $\mathbb{F}_2$ so 


$$v_1 + v_2 + v_3 = 0 \iff v_1 + v_2 + (v_3 + v_3) = v_3 \iff v_1 + v_2 = v_3 $$

And in the same way $v_1 + v_2 = v_4$. This implies that
$$v_3 + v_4 = (v_1 + v_2 )+ (v_1 + v_2) = 0$$
so the set $\{v_3, v_4\}$ is linearly dependent which is a contradiction. So $U_{2,4}$ is not $\mathbb{F}_2$-representable. However $U_{2,4}$ is $\mathbb{F}_3$-representable, for instance the following matrix taken from \cite[20]{oxley1} works

$$A = \begin{pmatrix}
    1 & 0 & 1 & 1 \\
    0 & 1 & 1 & 2
\end{pmatrix}$$
since we see that no vector is 0, so all singletons are independent. No vector is a scalar multiple of each other, so two element subsets are independent. Finally, no 3-element subsets can be independent, since the dimension of the $\mathbb{F}_3^2$ is 2 so dimension of any subspace is at most 2.

In the succeding sections we will also consider an example of a matroid, which is not representable over any field.