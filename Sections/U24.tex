

We will consider the uniform matroid $U_{2,4}$ and show that it is $\mathbb{F}_3$-representable, while it is not $\mathbb{F}_2$-representable. First, we note that $U_{2,4}= (\{1,2,3,4\}, \mathcal{I})$ where the bases are all of the two element subsets of $\{1,2,3,4\}$. So the matroid looks like: 



\begin{figure}[h]\label{u24}

\begin{center}

\begin{tikzpicture}

\matrix (a) [matrix of math nodes, column sep=0.6cm, row sep=0.6cm,]{
 & & &\textcolor{cyan}{
1234} & & & &\\
 \textcolor{blue}{
123}& &\textcolor{blue}{
124} & &\textcolor{blue}{
134} &  & \textcolor{blue}{
234}  \\
\textcolor{red}{12} & \textcolor{red}{13} & \textcolor{red}{14} & & \textcolor{red}{23} & \textcolor{red}{
24} & \textcolor{red}{
34} \\
\textcolor{orange}{1}& &\textcolor{orange}{2} & & \textcolor{orange}{3}& & \textcolor{orange}{4} \\
& & & \textcolor{orange}{\emptyset} &  & & \\
&&&&&& \\};

\foreach \i/\j in {1-4/2-1, 1-4/2-3, 1-4/2-5, 1-4/2-7, 2-1/3-1, 2-1/3-2, 2-1/3-5, 2-3/3-1, 2-3/3-3, 2-3/3-6, 2-5/3-2, 2-5/3-3, 2-5/3-7, 2-7/3-5, 2-7/3-6, 2-7/3-7, 3-1/4-1, 3-1/4-3, 3-2/4-1, 3-2/4-5, 3-3/4-1, 3-3/4-7, 3-5/4-3, 3-5/4-5, 3-6/4-3, 3-6/4-7, 3-7/4-7, 3-7/4-5, 4-1/5-4, 4-3/5-4, 4-5/5-4, 4-7/5-4}
\draw[double, line width = 0.005mm, color = brown] (a-\i) -- (a-\j);

\node[draw] at (0, -2.5){\small \textcolor{orange}{Independent set}, \textcolor{red}{Basis}, \textcolor{cyan}{Dependent set}, \textcolor{blue}{Circuit} };

\end{tikzpicture}
\end{center}
\caption{Representation of $U_{2,4}$}

\end{figure}

Suppose $U_{2,4}$ were $\mathbb{F}_2$-representable. Then there is some $A = \mat_{m \times 4}(\mathbb{F}_2)$, 

$$A = \begin{pmatrix}
    v_1 & v_2 & v_3 & v_4
\end{pmatrix}$$

so that $U_{2,4} \sim M[A]$. We have the following observation.

\begin{lemma}
\label{f2lema}
    Suppose a set of vectors $A = \{w_1, w_2, \cdots, w_r\} \subseteq \mathbb{F}_2^n$ is minimaly linarly dependent, that means that it is linarly dependent but any proper subset of it is linearly independent. Then $w_1 + w_2 + \cdots + w_r = 0$.
\end{lemma}

\begin{proof}
    Because $A$ is linearly dependent, there exists a set of scalars $\{a_1, a_2, \cdots, a_r\}\in \mathbb{F}_2$ not all zero such that
    
    $$\sum_{i=1}^r a_iv_i = 0.$$
    
    If for some $1\leq j \leq r$ we have $a_j = 0$ then we also have 
    
    $$\sum\limits_{\substack{i = 1 \\ i \neq j}} ^r a_iv_i = 0$$

    but not all out of $a_1, \cdots a_{j-1}, a_{j+1}, \cdots a_r$ are 0. This means that ${v_1, \cdots, v_{j-1}, v_{j+1}, \cdots, v_r}$ is linearly dependent, which is not true by assumption. Therefore, for all $1\leq j \leq r$ we have $a_j \neq 0$. But since the field is $\mathbb{F}_2$ this forces $a_j = 1$ for all $j$. Hence, we obtained what we want, namely

     $$\sum_{i=1}^r a_iv_i = 0 =  \sum_{i=1}^r 1 \cdot v_i = 0.$$
    
\end{proof}

If the matroid $U_{2,4} \sim M[A]$ for the above $A$ then we would have all of the three element subsets of vectors to be minimally linearly dependent. This means by lemma (\ref{f2lema}) that $v_1 + v_2 + v_3 = 0$ and $v_1 + v_2 + v_3 = 0$. However, we are in $\mathbb{F}_2$ so 


$$v_1 + v_2 + v_3 = 0 \iff v_1 + v_2 + (v_3 + v_3) = v_3 \iff v_1 + v_2 = v_3 $$

And in the same way $v_1 + v_2 = v_4$. This implies that
$$v_3 + v_4 = (v_1 + v_2 )+ (v_1 + v_2) = 0$$
so the set $\{v_3, v_4\}$ is linearly dependent, which is a contradiction. So $U_{2,4}$ is not $\mathbb{F}_2$-representable. However, $U_{2,4}$ is $\mathbb{F}_3$-representable. For instance the following matrix taken from \cite[20]{oxley1} works, namely

$$A = \begin{pmatrix}
    1 & 0 & 1 & 1 \\
    0 & 1 & 1 & 2
\end{pmatrix}$$.

 We see that no column vector of $A$ is 0, so all single-element subsets are independent. No vector is a scalar multiple of each other, so two element subsets are independent. Finally, no 3-element subsets can be independent, since the dimension of the $\mathbb{F}_3^2$ is 2, so the dimension of any subspace is at most 2.

To conclude the example of $U_{2,4}$, we will show that $U_{2,4}$ has an interesting property that it is not a graphic matroid. In other words, we can show that there is no graph $G$ such that $U_{2,4} \sim M$ where $M$ is the matroid formed by the set of edgs of $G$ with the usual rules. 

We will show $U_{2,4}$ is not graphic by contradiction, that is assume $U_{2,4}$ is graphic and let $G$ be the graph it represents. Then $G$ has four edges and since all two-element subsets of $U_{2,4}$ are independent we see that $G$ has no loops or parallel edges. In particular, all three-element subsets of $U_{2,4}$ are circuits, so in the graph the edges they correspond to form a cycle. But if both sets $\{1,2,3,\}$ and $\{1,2,4\}$ corresonds to edge cycles without parallel edegs it immediatly follows that $3$ and $4$ connect the same vertices, i.e. are parallel edges, which is a contradiction. It follows that $U_{2,4}$ is not graphic.

%In the succeeding sections, we will also consider an example of a matroid, which is not representable over any field.
