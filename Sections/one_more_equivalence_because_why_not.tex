\newpage

\section{Flats}

An excercise in \cite[35]{oxley1} asks us to prove a cryptomorphic description of matroids in terms of their flats. We think it is a prototypical model of a statement of the form (a collection of subsets satisfies properties X) if and only if (it is some well known family of matroid subsets), hence we will give our proof of the statement.

    Given a matroid $M$ with a ground set $E$ and its collection of flats, that is, all the subsets $X \subseteq E$ satisfying $\cl(X) = X$, the statement 'abstracts out' which properties of flats \textit{make them flats}. Such statements are a common theme of this article.

\begin{theorem}
    Let $M = (E, \mathcal{I})$ be a matroid. A collection $\caf$ is a collection of flats if and only if the following three conditions hold

    \begin{enumerate}
        \item[(F1)] We have $E \in \caf$.
        \item[(F2)] If $F_1, F_2 \in \caf$ then $F_1 \cap F_2 \in \caf$.
        \item[(F3)] If $F \in \caf$ and $\{F_1, F_2, \cdots, F_k\}$ is a collection of all minimal members of $\caf$ properly containing $F$ then $\{F_1-F, F_2-F, \cdots, F_k - F\}$ partition $E-F$. By defintion of partition that means that for all $i \neq j$ we have $(F_i-F)\cup (F_j - F)= \emptyset$ and $\cup_{i = 1}^{k}(F_i - F) = E - F$.
        
    \end{enumerate}
        
\end{theorem}

As an example, we can once again return to our depiction from earlier. This time, we write flats in cursive and with a line below it. As one can see, every superset of a flat has a larger rank. One can see here that the ground set is indeed a flat, and the intersection of for example \textit{134} and \textit{24} is indeed another flat, \textit{4}. In this picture, only dependent sets are flats. However, if all supersets one larger than a certain independent set are also independent, then that independent set is a flat. We unfortunately do not have such a situation in this matroid. 

\begin{figure} [H]
\begin{center}
\begin{tikzpicture}

\matrix (a) [matrix of math nodes, column sep=0.6cm, row sep=0.6cm,]{
 & & & \textcolor{cyan}{\underline{\textit{1234}}}_2 & & & &\\
 \textcolor{cyan}{
123}_2& &\textcolor{cyan}{
124}_2 & &\textcolor{cyan}{\underline{\textit{134}}}_1 &  & \textcolor{cyan}{
234}_2  \\
\textcolor{red}{12}_2 & \textcolor{blue}{13}_1 & \textcolor{cyan}{14}_1 & & \textcolor{red}{23}_2 & \textcolor{cyan}{\underline{\textit{24}}}_1 & \textcolor{cyan}{
34}_1 \\
\textcolor{orange}{1}_1& &\textcolor{orange}{2}_1 & & \textcolor{orange}{3}_1& & \textcolor{blue}{\underline{\textit{4}}}_0 \\
& & & \textcolor{orange}{\emptyset}_0 &  & & \\
&&&&&& \\};

\foreach \i/\j in {1-4/2-1, 1-4/2-3, 1-4/2-5, 1-4/2-7, 2-1/3-1, 2-1/3-2, 2-1/3-5, 2-3/3-1, 2-3/3-3, 2-3/3-6, 2-5/3-2, 2-5/3-3, 2-5/3-7, 2-7/3-5, 2-7/3-6, 2-7/3-7, 3-1/4-1, 3-1/4-3, 3-2/4-1, 3-2/4-5, 3-3/4-1, 3-3/4-7, 3-5/4-3, 3-5/4-5, 3-6/4-3, 3-6/4-7, 3-7/4-7, 3-7/4-5, 4-1/5-4, 4-3/5-4, 4-5/5-4, 4-7/5-4}
\draw[double, line width = 0.005mm, color = brown] (a-\i) -- (a-\j);

\node[draw] at (0, -3){\small \textcolor{orange}{Independent set}, \textcolor{red}{Basis}, \textcolor{cyan}{Dependent set}, \textcolor{blue}{Circuit}, $\text{Set}_{\rank(\text{Set})}$, \underline{\textit{Flat}} };

\end{tikzpicture}
\end{center}

    \caption{Description of a 4-element matroid, with the flats in cursive and with a line below it.}
    \label{fig:enter-label}
\end{figure}


\begin{proof}
Before we begin the proof, we have two notable observations. First, we see that the characterization of our collection $\caf$ follows a similar pattern as characterization of other encountered collections, such as $\mathcal{I}$ or $\mathcal{C}$. Specifically, we mean that we have two initializing properties, for example, the collection is non-empty, has $\emptyset$ or $E$, is downward closed, closed under intersections, inclusions, etc. Followed by the third property, which really tells us something non-trivial about how the sets of the collection interact with each other. 
Second, as with the previous proofs, we will see that it is not hard to prove the only if direction, i.e. knowing we have a collection of flats and proceeding further. However, we will require some clever ideas to prove the reverse direction.

\begin{enumerate}
     
\item[$\implies$]
So first suppose $\caf$ is a collection of flats and our goal is to prove that the three properties hold. 

\begin{enumerate}
    \item[(F1)] We see that $E \subseteq \cl(E)$ by the definition of the closure operator and $\cl(E)\subseteq E$ because $\cl$ maps from $2^E$ into $E.$ Thus $\cl(E)= E$ implying $E$ is a flat as desired.
    
    \item[(F2)] Again, inclusion $F_1 \cap F_2 \subseteq \cl(F_1 \cap F_2)$ is a fundamental property of closure operator (CL1). Because $F_1 \cap F_2 \subseteq F_1$ we have by (CL2) that $\cl(F_1 \cap F_2) \subseteq \cl(F_1) = F_1$ where the last equality follows by assumption that $F_2$ is a flat. Similarly, $\cl(F_1\cap F_2) \subseteq F_2$, which implies $\cl(F_1\cap F_2)$ is contained in $F_1$ and $F_2$, in other words $\cl(F_1 \cap F_2) \subseteq F_1 \cap F_2$. To conclude $F_1 \cap F_2 \subseteq \cl(F_1 \subseteq F_2) \subseteq F_1 \cap F_2$ which implies $\cl(F_1 \cap F_2) = F_1 \cap F_2$, and then by definition $F_1 \cap F_2 \in \caf$ thus the second property is satisfied.

    \item[(F3)] We pick a flat $F$ and let $P = \{F_1, F_2, \cdots, F_k\}$ be the collection of all flats properly containing $F$. To show that $P$ partitions $E - F$ we have to show two things; for any $i \neq j$ we must have $(F_i - F)\cap( F_j - F) = \emptyset$ and $\cup_{k}( F_k - F) = E - F.$ 
    
    The second property is evident, namely if we pick any $x \in E - F$ then there exists a flat containing $F \cup x, $ in particular $E$ will do. So there exists a minimal (not properly containing others with such properties) flat containing $F \cup x$, let us call it $G$ and obviously, $G = \cl(F \cup x)$. We would like $G$ to also be a minimal flat properly containing $F$. If it is not, there is some flat $H$ such that $F \subseteq H \subseteq G$ where both inclusions are proper. By assumption $H$ does not include $x$ otherwise it would contradict the minimality of $G.$ But there also has to be some $y \in H - F$. In particular, since all things involved are flats, we have that $y \in \cl(F \cup x) - \cl(F)$ so by (CL4) we have $x \in \cl(F \cup y)$ which is a contradiction since $\cl(F \cup y)$ is a flat which contains $F$ and $y$, so it is $H.$  Therefore $G$ is a minimal flat and we are done.

    The first property can be derived by contradiction. Namely, if for some $i\neq j$ we have that there exists $e \in (F_i - F)\cup(F_j - F)$ then we have that the intersection $F_i \cup F_j$ is a flat which contains $F\cup e$, but is also properly contained inside $F_j$ and $F_i$ - a contradiction.

    So we also have the last property.
    
\end{enumerate}

\item[\impliedby]
    Now for the if direction. In the previous proofs of proving the reverse direction of such statements we could directly relate the independent sets, which we consider to be our most elementary definition, to our objects. For example, if you know your collection has to be the collection of circuits, then the independent sets are precisely all the proper subsets of circuits. If you know your collection has to be a collection of bases, then the independent sets have to be all of their subsets. When proving the equivalence of the rank function, the independent sets were precisely the sets with the property $r(X) = |X|$. 

    However, with flats, there is no way (at least not as clear as with previous examples) to relate flats with independent sets. So we will prove the reverse direction by relating it to one definition we already know. So we have to prove that $f$ satisfies the properties (CL1) - (CL4).

    Suppose we are given a finite set $E$ and collection $\caf$ of its subsets satisfying properties 1., 2. and 3. Then let us define a \textit{function} $f: 2^E \to \caf$ by the rule that for any $X$, we have $f(X)$ is the minimal member of $\caf$ containing $X$. It is clear what is our aim, namely to show that this is in fact a closure operator, then we will do two things at once - $E$ will be a matroid with a closure operator $f$ and the elements of $\caf$ will be precisely the flats.

    First, we have to check $f$ is well-defined. By the first property of $\caf$ we have that $E \in \caf$ and because for any subset $X \subseteq E$ we have $X \subseteq E$ we have that for every subset $X$ there \textit{exists} some element of $\caf$ such that $X$ is a subset of it (in the worst case it is $E$). Now, the minimal such subset has to be unique, since if $F_1, F_2$ have the same property that they are minimal members of $\caf$ including $X$ then by the second property of $\caf$ we have that $F_1 \cap F_2 \in \caf$ and so $X \subseteq F_1\cap F_2$, so if they are not the same we are contradicting the minimality of them. Therefore $ F_1 \subseteq F_2$ and $F_2 \subseteq F_1$ hold, so $F_1 = F_2.$


\begin{enumerate}
    \item[(CL1)] If $X \subseteq E$ then by definition, $f(X)$ is an element of $\caf$ \textit{containing} $X$, so we have $X \subseteq f(X)$ and the first property follows.
    
    \item[(CL2)] If $X \subseteq Y \subseteq E$ then $f(X)$ is the minimal member of $\caf$ containing $X$. Because $f(Y)$ is a minimal member of $\caf$ which contains $Y$ it thus also contains $X$ so by minimality of $f(X)$ we have $f(X)\subseteq f(Y)$ as desired.

    \item[(CL3)] If $X \subseteq E$ is arbitrary then we have that $f(f(X))$ is the minimal member of $\caf$ containing $f(X)$, but $f(X)$ is by definition itself in $\caf$, and it contains $f(X)$. So we have that $f(X) \subseteq f(f(X)) \subseteq f(X)$ implying $f(f(X)) = f(X).$

    \item[(CL4)] Up to this point we have not yet used the second and the third property of $\caf$ so we better require them now.

     Suppose $X \subseteq E$, $x \in E$ and $y \in f(X \cup x) - f(X)$. In particular, this means that $y \notin X$.
   Our goal is to show $x \in f(X \cup y)$. Let us call $f(X) = F_1$, $f(X \cup x) = F_2$ and $f(X \cup y) = F_3$ to remind us that output of $f$ is always a member of $\caf$. By the second property of "closure operator" we have already proved, we have $F_1 \subseteq F_2$ and $F_1 \subseteq F_3$. Not just that, $F_2$ and $F_3$ are in fact one of the minimal elements of $\caf$ properly containing $F_1$.

     We know this, because by the third property of $\caf$, all of the minimal elements of $\caf$ containing $F_1$ \textit{partition} $E-F_1$, so in particular, there is some $F_2' \in \caf$ which contains $X\cup x$ and contains $X$ minimally and some $F_3' \in \caf$ which contains $X \cup y$ and contains $X$ minimally. So from $X \cup x \subseteq F_2' \in \caf$ we directly infer $F_2 \subseteq F_2'$ because $f(X \cup x)$ is an element of $\caf$ which contains $X \cup x$ minimally. It is now clear that $F_2= F_2'$ and $F_3 = F_3'$.


     We are almost done. By the third property of $\caf$, because $F_2$ and $F_3$ are minimal members of $\caf$ properly containing $X$ we have that $(F_2-F_1)\cap (F_3 - F_1)$ is either empty or $F_2 = F_3$. It cannot be empty because, $y \in F_3 - F_1$ by definition and $y \in f(X\cup x)-f(X) = F_2 - F_1$ by assumption. So $F_2 = F_3$ which implies that $x \in F_3 = f(X \cup y).$ This shows the last closure axiom, and in particular, we know that we have a matroid with ground set $E$ and closure operator $f.$


     Finally, it is clear by definition that the collection of all subsets of $X$ such that $f(X) = X$ is precisely $\caf$, so $\caf$ is our collection of flats for this matroid.
     
\end{enumerate}

\end{enumerate}
    
\end{proof}
