\subsection{Fields}\label{sec:appendix-fields}

This section provides a basic introduction to working with fields. For a more formal in depth treatment of the topic, check~\cite{milne2022}.

\begin{defn}[Fields]
  Intuitively, a field is a triplet consisting of a set and two binary operations, which are usually referred to as addition and multiplication. Both these operations are associative, commutative, and have identity elements (usually referred to as $0$ and $1$ respectively). All elements have inverse elements for addition. Similarly, all non-zero elements have inverses for multiplication. Furthermore, the two operations obey distributivity laws.
\end{defn}

\begin{defn}[Subfields]
  Intuitively, given a field $\mathbb{K}$, a subfield is a subset $\mathbb{F} \subseteq \mathbb{K}$ together with restrictions of addition and multiplication to $\mathbb{F}$. Formally, the following three properties must hold:
  \begin{enumerate}
    \item We have $1 \in \mathbb{F}$.
    \item If $a, b \in \mathbb{F} $ then $ a - b \in \mathbb{F}$.
    \item If $a, b \in \mathbb{F} \;\text{and}\; b \neq 0$ then $ a b ^{-1}  \in \mathbb{F}$
  \end{enumerate}
\end{defn}

\begin{defn}[Algebraic elements]
  Given a field $\mathbb{K}$ and a subfield $\mathbb{F}$, an element $a \in \mathbb{K}$ is said to be algebraic over $\mathbb{F}$ if there exists a non-zero polynomial $f$ with coefficients in $\mathbb{F}$ such that $f(a) = 0$. Elements which are not algebraic are said to be transcendental.
\end{defn}


Throughout \hyperref[sec:algebraic-matroids]{the section on algebraic matroids} we use the notation $\mathbb{F}[X _1 \cdots X_n]$ to refer to the set of polynomials with $n$ variables and coefficients in some field $\mathbb{F}$. This set presents a ring structure, but understanding that is not necessary for understanding this article. A more in depth treatment of polynomial rings can be found in~\cite{milne2022}.


\begin{defn}[Adjoining elements to a field] \label{def:field-adjoint}
  Given a field $\mathbb{F}$ and some element $e$, we can construct a new field $\mathbb{F}(e)$ with elements of the form:
\begin{align*}
  \frac{\sum a _i e ^i }{\sum b _i e ^i },
\end{align*}
where $a _i, b _i \in \mathbb{F}$.

  Furthermore, we will use the notation $\mathbb{F}(a _1 ,\cdots, a _n)$ to refer to $\textbf{F}(a _1) \cdots (a _n)$.
\end{defn}

\newpage
\begin{exmp} \label{exp:q-adjoint-sqrt-3}
  Consider the field $\mathbb Q$  and the element $\sqrt 3$. We notice that 
  \begin{align*}
    {(\sqrt 3)} ^{2k} = 3 ^k && \text{and} && {(\sqrt 3)} ^{2k + 1} = 3 ^k \sqrt 3.
  \end{align*}

  This let us rewrite the form the elements are introduced in by definition (\ref{def:field-adjoint}) as
  \begin{align*}
    \frac{\sum a _i {(\sqrt 3)} ^i }{\sum b _i {(\sqrt 3)} ^i }
    &= \frac{\sum a _{2i} 3 ^i + \sum a_{2i + 1} 3 ^i \sqrt 3}{\sum b _{2i} 3 ^i + \sum b_{2i + 1} 3 ^i \sqrt 3} 
    \\&= \frac{a + b\sqrt 3}{c + d\sqrt 3}
    \\&= \frac{(a + b \sqrt 3)(c - d \sqrt 3)}{(c + d \sqrt 3)(c - d \sqrt 3)}
    \\&= \frac{(ac - 3bd) + (bc - da) \sqrt 3}{c ^2 - 3 d ^2 }
    \\&= \frac{ac - 3bd}{c ^2 - 3 d ^2 } + \frac{bc - da}{c ^2 - 3 d ^2 } \sqrt 3,
  \end{align*}
  where the second equality follows by choosing rationals $a, b, c, d \in \mathbb{Q}$ defined as
  \begin{align*}
    a = \sum a _{2i} 3 ^i && b = \sum a _{2i + 1} 3 ^i  &&
    c = \sum b _{2i} 3 ^i && d = \sum b _{2i + 1} 3 ^i 
  \end{align*}

  We can now define $p = \frac{ac - 3bd}{c ^2 - 3 d ^2 }$ and $q = \frac{bc - da}{c ^2 - 3 d ^2 }$ to show that all elements of $ \mathbb{Q} (\sqrt{3})$ are of the form $p + q \sqrt 3$. Moreover, it follows trivially from  definition (\ref{def:field-adjoint}) that all elements of the form $p + q \sqrt 3$ are elements of $\mathbb{Q} (\sqrt{3})$. Let $\mathbb{F} = \{p + q \sqrt 3 \:|\: p, q \in \mathbb{Q} \}$ Having proven both 
  \begin{align*}
     \mathbb{F} \subseteq \mathbb{Q} (\sqrt 3) && \text{and} && \mathbb{F} \supseteq \mathbb{Q} (\sqrt 3),
  \end{align*}
 we can conclude that $\mathbb{Q} (\sqrt{3})  = \mathbb{F}$.

  Addition will then be performed component-wise, i.e. 
  \begin{align*}
  (a + b \sqrt 3) + (c + d \sqrt 3) = (a + c)  + (b + d) \sqrt 3.
  \end{align*}

  Multiplication is a bit more involved:
  \begin{align*}
    (a + b \sqrt 3)(c + d \sqrt 3) = (ac + 3bd) + (ad + bc) \sqrt 3.
  \end{align*}

  We will now prove that $ \mathbb{Q}(\sqrt{3})$ is indeed a field, as stipulated by the definition. Associativity,  commutativity, and distributivity follow by direct computation. Similarly, $0$ and $1$ are both of the form $p + q \sqrt 3$, so both additive and multiplicative identities exist ($a + 0 = a$ and $a \cdot 1 = a$ can also be checked by direct computation). Furthermore, for any $p + q \sqrt 3$ we have $-p -q \sqrt 3$ such that 
  \begin{align*}
    p + q\sqrt 3 + (- p - q\sqrt 3) = (p - p) + (q - q) \sqrt 3 = 0.
  \end{align*}

  Similarly, if $p, q \neq 0$ we notice that $p ^2 - 3 q ^2 = 0$ holds only if $p = q \sqrt 3$, but this is never the case since $p, q \in \mathbb{Q}$. We construct $\frac p {p ^2 - 3 q ^2 } - \frac q {p ^2 - 3 q ^2 } \sqrt 3$ such that
  \begin{align*}
    (p + q \sqrt 3)\left(\frac p {p ^2 - 3 q ^2 } - \frac q {p ^2 - 3 q ^2 } \sqrt 3\right)
    &= \frac {p ^2}  {p ^2 - 3 q ^2 } - \cancel{\frac {pq} {p ^2 - 3 q ^2 } \sqrt 3}
    + \cancel{\frac {pq}  {p ^2 - 3 q ^2 } \sqrt 3} - 3\frac {q ^2 } {p ^2 - 3 q ^2 }
    \\&= \frac{p ^2 - 3 q ^2 }{p ^2 - 3 q ^2 }
    \\&= 1.
  \end{align*}

  We can now conclude that all elements have additive inverses, and that all non-zero elements have multiplicative inverses. This, together with the previously proven properties, let us conclude that $\mathbb{Q} (\sqrt{3})$ is indeed a field.
\end{exmp}

\begin{defn}[Field extensions]
 Given a field $\mathbb{K}$ and a subfield $\mathbb{F}$, the field $\mathbb{K}$ is said to be a field extension of $\mathbb{F}$.
\end{defn}

\begin{lemma}[Vector space structure of field extensions]\label{lem:field-extension-vector-space}
  Given a field $\mathbb{F}$ and a field extension $\mathbb{K}$, we can construct a vector space with elements of $\mathbb{F}$ as scalars, elements of $\mathbb{K}$ as vectors:
  \begin{itemize}
    \item vector addition (together with its identity element and inverse) is defined the same way as addition in $\mathbb{K}$. 
    \item scalar-vector multiplication (together with its identity element and inverse) is defined the same way as multiplication in $\mathbb{K}$ (because all scalars are also vectors, since elements of $\mathbb{F}$ are also elements of $\mathbb{K}$).
    \item scalar-scalar multiplication (together with its identity element and inverse) is defined the same way as multiplication in $\mathbb{F}$.
  \end{itemize}
\end{lemma}

\begin{lemma}\label{lem:algebraic-field-finite-vector-space}
   Given a field $\mathbb{K}$ and a subfield $\mathbb{F}$, with some $a \in \mathbb{K}$ algebraic over $\mathbb{F}$, the vector space induced by $\mathbb{F}(a) $ over $\mathbb{F}$ will be finite-dimenssional.
\end{lemma}

Proving the above lemma is outside the scope of this article, but a proof can be found in~\cite{milne2022}.

\begin{exmp}
  We will study the vector space induced by $\mathbb{Q} (\sqrt 3)$ over $\mathbb{Q}$. We recall from example (\ref{exp:q-adjoint-sqrt-3}) that elements of $\mathbb{Q} (\sqrt 3)$ are of the form $p + q \sqrt 3$ where $p, q \in \mathbb{Q}$. We notice that $B = \{1, \sqrt 3\}$ forms a basis for the vector space induced by $\mathbb{Q} (\sqrt 3)$ over $\mathbb{Q} $. It's clear that $\spann B = \mathbb{Q} (\sqrt 3)$ (because for any $v = p + q \sqrt 3 \in \mathbb{Q} (\sqrt 3)$ we can pick scalars $p, q \in \mathbb{Q} $ such that $p \cdot 1 + q \cdot \sqrt 3 = v$). 

  It remains to show that $1$ and $\sqrt 3$ are linearly independent. Assume this is not the case. We now have non-zero $p, q \in \mathbb{Q} $ such that $p + q \sqrt 3 = 0$. If $q = 0$ then $p = 0$, hence a contradiction. Some simple algebraic manipulations then imply that $\sqrt 3  = \frac p {-q}$, but $\frac p {-q} \in \mathbb{Q} $ and $\sqrt 3 \not\in \mathbb{Q} $, hence another contradiction.
\end{exmp}
