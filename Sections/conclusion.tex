\section{Conclusion}
In the article we discussed several equivalent definitions of matroids. As a starting point we choose the definition in terms of independent sets. We saw they naturally abstract the linear indepenent subsets of vector spaces. We then proceeded by showing the description in terms of maximal independent sets - bases. Afterwards we moved to circuits and how they can on the one hand be interpreted as minimal dependent or as object naturally abstracting graphic cycles on the other. We then moved on to more intricate objects which described the matroid by assining some value to each set. First we saw the rank function and second the closure operator, both providing a new perspective on how to describe the matroids by not listing subsets directly, but rather present it using a fucntion. Lastly, we introduced the concept of flat. Along the way we proved the cryptomorphism - that is that the matroid can be described starting from any viewpoint. As an example we showed that the algebraically dependent sets in field extensions also have a natural matroid structure, even though the matroid modeled a much broader concept of algebraic independence and no longer linear independence. As our last illustration of matroids we turned away from cryptomorphisms and defined the concept of a matroid dual which had many interesting properties. Finally we showed that a dual of a graphic matroid can be non-graphic.
