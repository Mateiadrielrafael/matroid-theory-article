\newpage
\section{Conclusion}
In the article, we discussed several equivalent definitions of matroids. As a starting point, we choose the definition in terms of independent sets. We saw they naturally abstract the linear independent subsets of vector spaces. We then proceeded by showing the description in terms of maximal independent sets - bases. Afterward we moved to circuits and how they can, in one hand, be interpreted as minimal dependent, or on the other, as object naturally abstracting graphic cycles. We then moved on to more intricate objects, which described the matroid by assigning some value to each set. First, we saw the rank function and second the closure operator, both providing a new perspective on how to describe the matroids by not listing subsets directly, but rather present it using a function. Lastly, we introduced the concept of flat. Along the way, we proved the cryptomorphism - that is, that the matroid can be described starting many different viewpoints. As an example, we showed that the algebraically dependent sets in field extensions also have a natural matroid structure, even though the matroid modeled a much broader concept of algebraic independence and no longer linear independence. With all of this, we achieved our objective of answering our guiding question - \textbf{how can we define matroids?} As our last illustration of matroids we turned away from cryptomorphisms and defined the concept of a matroid dual which had many interesting new properties, and also extended the existing ones, such as it was the case with the corank function, coindependent sets, cobases and cocircuits. Finally, we showed that it is possible for the dual of a graphic matroid to be non-graphic.
