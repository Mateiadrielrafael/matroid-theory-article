\subsection{Partially ordered sets}\label{sec:appendix-poset}

\begin{defn}[Partially ordered set]
 A partially ordered set is a pair containing a set $S$ and a partial order $\leq$. A partial order is a binary relation on $S$ with the following properties:
 \begin{itemize}
   \item For all $ x \in S$ we have $ x \leq x$ (reflexivity)
   \item For all $a, b \in S$ if $ a \leq b $ and $ b \leq a $ then $a = b$ (anti-symmetry)
   \item For all $ a, b, c \in S, $ if $a \leq b $ and $b \leq c$ then $ a \leq c$ (transitivity)
 \end{itemize}
 \end{defn}

 \begin{exmp}
   Given a set $U$, inclusion induces a partial order $(S, \subseteq )$ on any subset $S \subseteq 2 ^{U}$. All three properties follow directly from the definition of $\subseteq $.
 \end{exmp}

 \begin{defn}[Maximal elements]
   Given a partially ordered set $(S, \leq )$ an element $M \in S$ is said to be maximal if for all $c \in S$ such that $c\leq M$ we have $c = M.$


 \end{defn}

 \begin{exmp}\label{exp:poset-maximal-element}
   Consider the partially ordered set induced by $\subseteq$ on the subsets of $U = \{1, 2, 3\}$. It can be checked by cases that $\{1, 2, 3\}$ is the only maximal element on $(2 ^U, \subseteq )$.
 \end{exmp}


 \begin{remark}\label{rem:subsets-of-poset-are-posets}
   Given a partially ordered set $(U, \leq )$ and a subset $S \subseteq U$, we can define a new partially ordered set $(S, \leq )$ by restricting $\leq$ to $S$.
 \end{remark}

 \begin{exmp}
   Consider the partially ordered set introduced in example (\ref{exp:poset-maximal-element}). We can remove $\{1, 2, 3\}$ to form a new partially ordered set $(S, \subseteq )$ where $S = 2 ^U - \{1, 2, 3\}$. We now have multiple maximal elements - they are $\{1, 2\}, \{1, 3\}$ and $\{2, 3\}.$
 \end{exmp}

 \begin{lemma}\label{lem:non-empty-finite-posets-have-maximal-elements}
   Every non-empty finite partially ordered set $(S, \leq )$ has at least one maximal element. 
 \end{lemma}

 \begin{proof}
  We will prove the statement by induction on the size of $S$:
  \begin{itemize}
    \item If $S = \{e\}$ then $e$ is a maximal element.
    \item Assume the statement holds for all sets $S$ such that $|S| = n$. We will attempt to prove it holds for $|S| = n + 1$ as well. 

      Consider some arbitrary $s \in S$. By the induction hypothesis, $(S - s, \leq)$ (see remark (\ref{rem:subsets-of-poset-are-posets})) has a maximal element $M$. We will proceed by cases:
      \begin{enumerate}
        \item If $M$ is also a maximal element for $(S, \leq)$, we are done.
        \item Otherwise, there must exist some $c \in S - M$ such that $M \leq  c$. If $c \in S - s$ we have a contradiction (because $M$ wouldn't be a valid maximal element for $(S - s, \leq )$). It then follows that $c = s$. 

          We claim that $c$ is a maximal element. Consider $d \in S$ satisfying $c \leq d$. If $d = c$ we are done. Otherwise, $d \in S - s$, so $d \leq M$ ($M$ is maximal for $(S - s, \leq )$), and by transitivity with $M \leq c$ we have $d \leq c$, which by anti-symmetry implies $d = c$.
      \end{enumerate}
  \end{itemize}
 \end{proof}
