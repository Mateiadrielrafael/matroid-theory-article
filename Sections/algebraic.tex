\section{Algebraic matroids}

An interesting concept of independence arises in the study of field theory. We will present an interesting class of matroids derived from the concept of algebraic independence. The topics discussed here are furthered elaborated on in~\cite{milne2022}. We start by introducing a lemma that's crucial for the following proofs, and it's prerequisites.

\begin{defn}
    A partially ordered set is a pair containing a set $S$ and a reflexive anti-symmetric transitive relation $\leq$. A totally ordered set is a partially ordered set for which $\leq$ is total, i.e.\ $\forall x, y \in S, x \leq y \lor y \leq x$.
\end{defn}

\begin{defn}
    An element $m \in S$ is said to be maximal with respect to a partially ordered set $(S, \leq)$ if $\forall s \in S, s \geq m \implies s = m$.
\end{defn}

\begin{defn}
    An element $u$ is said to be an upper bound for a subset $S$ of a partially ordered set if $\forall s \in S, u \geq s$.
\end{defn}

\begin{lemma}[Zorn]\label{lem:Zorn}
   Given a partially ordered set $P$, with the property that any totally ordered subset has an upper bound in $P$, then a maximal element for $P$ exists.
\end{lemma}

This lemma is equivalent to the axiom of choice, and will be assumed for the rest of the section.
