\section{Algebraic matroids}\label{sec:algebraic-matroids}

An interesting concept of independence arises in the study of field theory. We will present a class of matroids derived from the concept of algebraic independence. For the rest of this section, we will use the notation $\mathbb K / \mathbb F$ to refer to some field $\mathbb K$ and a subfield $\mathbb F$. See \hyperref[sec:appendix-fields]{the appendix} for an introduction to fields and subfields.

\begin{defn}
	A subset $S = \{ s_1 \ldots s _n \} \subseteq \mathbb K / \mathbb F$ is said to be algebraically independent (over $\mathbb F$) if there does not exist any
	nonzero multivariable polynomial in $n$ variables $f \in \mathbb{F}[X _1 \ldots X _n]$, with coefficients in $\mathbb{F} $, such that the evaluation homomorphism $\ev_{(s_1, s_2,\cdots,s_n)}(f) = 0$.
	%\begin{align*}
	%\text{for all } f \in \mathbb{F}[X _1 \ldots X _n], \left( f \neq  0 \land  f(s _1 \ldots s _n) = 0\right).
	%\end{align*}
\end{defn}
Intuitively, the definition says that no linear combination with coefficients in $\mathbb F$ of products of elements in $S$ is equal to $0$.

We know that an element $x\in \mathbb{R}$ is called transcendental with respect to $\mathbb{Q}$ if for nonzero polynomials with rational coefficients $\ev_{x}(p) \neq 0$. That means that any nontrivial polynomial evaluated at $x$ will never give 0.
Comparing with the definition of algebraically independent sets above, we see that a singleton ${x}$ is algebraically independent by our defintion, if for any nonzero  single variable ($n = 1$) polynomial $p$ with coefficients in $\mathbb{Q}$ we will have that $\ev_{x}(p) \neq 0$. This is precisely the definition of transcendental numbers.

Although it is well known in the mathematical community that the complex numbers $e$ and $\pi$ are transcendental over $\mathbb{Q}$, it hasn't yet been proven that $\{e, \pi \}$ is algebraically independent.

We will now discuss the notion of algebraic dependence.

\begin{defn}
	Let $a \in \mathbb K / \mathbb F$ and $B = \{b _1, \ldots, b _n \} \subseteq  \mathbb K / \mathbb F$. We say that $a$ is algebraically dependent on $B$ (over $ \mathbb{F} $) if $a$ is algebraic over $\mathbb F(b _1, \ldots, b _n )$. Given some $A \subseteq  \mathbb K / \mathbb{F} $, we say that $A$ is algebraically dependent on $B$ (over $\mathbb{F} $) if every element of $A$ if algebraic on $B$ (over $\mathbb{F} $).
\end{defn}

Before continuing onto the next lemma, we remind the reader that a field $\mathbb{K}$ has a natural $\mathbb{K}$-vector space structure, thus scalars are elements of $\mathbb{F} $ and vectors being elements of $\mathbb K$. Moreover, the vector space induced by $\mathbb{F} (a)$ over $\mathbb{F} $ when $a$ is algebraic over $\mathbb{F} $ is finite dimensional. For more details, see lemmas (\ref{lem:field-extension-vector-space}) and (\ref{lem:algebraic-field-finite-vector-space}).

\begin{lemma}\label{lem:algebraic-transitivity}
	Consider $a, b \in \mathbb K / \mathbb{F} $ such that $a$ is algebraic over $\mathbb{F}(b) $ and $b$ is algebraic over $\mathbb{F}$. It then follows that $a$ is algebraic over $\mathbb{F}$.
\end{lemma}

\begin{proof}
	The vector spaces induced by $\mathbb{F} (b)$ and $\mathbb{F} (b)(a)$ must both be finite dimensional over $\mathbb{F} $ and $\mathbb{F} (b)$ respectively (because $b$ is algebraic over $\mathbb{F} $ and $a$ is algebraic over $\mathbb{F} (b)$). It follows that $\mathbb{F} (b)(a)$ is finitely dimensional over $\mathbb{F}$ (this can be seen by noticing that given a basis $\{a _i \}$ for $\mathbb{F} (b)(a)$ over $\mathbb{F} (b)$ and a basis $\{b _i\}$ for $\mathbb{F} (b)$ over $\mathbb{F} $, we can form a basis $\{a _i b _j\}$ for $\mathbb{F} (b)(a)$ over $\mathbb{F} $).

	Let $n$ be the dimension of $\mathbb{F} (b)(a)$ over $\mathbb{F} $. Let $S = \{a ^0 \ldots a ^n \}$. Because $|S| = n + 1$, $S$ cannot be linearly independent (in the vector space induced by $\mathbb{F} (b)(a)$ over $\mathbb{F} $), which means a polynomial $f \in \mathbb{F} [x]$ exists such that $ev_a(f) = 0$.
\end{proof}

\begin{lemma}\label{lem:algebraic-dependence-transitivity}
	Consider $A, B, C \subseteq \mathbb K / \mathbb{F} $ such that $A$ is algebraically dependent on $B$ and $B$ is algebraically dependent on $C$. It then follows that $A$ is algebraically dependent on $C$.
\end{lemma}

\begin{proof}
	We will proceed by induction on the size of $B$:
	\begin{enumerate}
		\item If $|B| = 1$ then the result trivially follows from lemma (\ref{lem:algebraic-transitivity}).
		\item Assume the result is true for $|B| = n$. We will attempt to prove it for $|B| = n + 1$. Let $a \in A$. We have to show that $a$ is algebraic over $\mathbb{F}(c _1 \ldots c _k )$. Given $1 \leq i \leq n$, each $b _i$ is algebraic over $\mathbb{F}(c _1 \ldots c _k )$ (because $B$ is algebraically dependent on $C$ (over $\mathbb{F}$)). It then follows that each such $b _i $ is also algebraic over $\mathbb{F}(c _1 \ldots c _k, b _{n + 1}) = \mathbb{F} (b _{n + 1} )(c _1 \ldots c _k )$. Moreover, we know that $a$ is algebraic over $\mathbb{F} (b _1 \ldots b _{n + 1})$, which means it is also algebraic over $\mathbb{F} (b _{n + 1})(b _1 \ldots b _n )$. It then follows from the induction hypothesis (applied with $\mathbb{F} (b _{n + 1})$ as the base field, $A = \{a\}$, $B = \{b _1 \ldots b _n \}$ and the same $C$) that $a$ is algebraic over $\mathbb{F} (b _{n + 1})(c _1 \ldots c _k ) = \mathbb{F} (c _1 \ldots c _k )(b _{n + 1})$.

		      We recall that $b _{n + 1}$ is algebraic on $\mathbb{F} (c _1 \ldots c _k )$ (because $B$ is algebraically dependent on $C$ (over $\mathbb{F} $)). We can now apply lemma (\ref{lem:algebraic-transitivity}) to prove that $a$ is algebraic over $\mathbb{F} (c _1 \ldots c _n)$.
	\end{enumerate}
\end{proof}

\begin{theorem}\label{thm:algebraic-matroids-are-matroids}
	Given a finite subset $E \subseteq \mathbb K / \mathbb F$, the pair $(E, \mathcal I)$  where $\mathcal I$ is the set of algebraically independent subsets of $E$ forms a matroid.
\end{theorem}

Although~\cite{oxley1} offers a proof based on independent sets and the concept of algebraic dependence, we will instead present a proof based on the bases of our supposed matroid. The part focused on proving lemma (\ref{lem:algebraic-indep-smaller-than-dep}) is inspired by~\cite{milne2022}, although the rest will be a lot simpler (no need to involve Zorn's lemma) because we are working in a finite subset.


We start by introducing some useful lemmas:
\begin{lemma}\label{lem:algebraic-indep-smaller-than-dep}
	Consider finite $A, B \subseteq \mathbb K / \mathbb F$, such that $A$ is independent over $\mathbb F$ and dependent on $B$ (over $\mathbb F$). Then $|B| \geq |A|$.
\end{lemma}

\begin{proof}
	We will prove the statement by induction on the number of elements in $A \setminus B$:
	\begin{enumerate}
		\item If $A \setminus B = \emptyset $ then $A \subseteq B$ and $|A| \leq |B|$ follows trivially.
		\item Assume the statemenet is true for some $|A \setminus B| = p$. We will attempt to prove it for $|A \setminus B| = p + 1$. Let $k = |A \cap B| = |A| - p$. Let $a _1 \ldots a _n$ be the elements of $A$ such that $a _1 \ldots a_k$ are all the members of $A \cap B$ for some $k$. Let $ a _1 \ldots a_k, b _{k + 1} \ldots b _m$ be the elements of $B$. It remains to prove that $m \geq n$.

		      As $a _{k + 1}$ is not dependent on $\{a _1 \ldots a_k\}$ but is dependent on $B = \{a _1 \ldots a _k, b _{k + 1} \ldots b_m\}$, there must exist some $k + 1 \leq j \leq m$ such that $a _{k + 1}$ is dependent on $\{a _1 \ldots a _k, b _{k + 1} \ldots b_j\}$ but  not on $\{a _1 \ldots a _k, b _{k + 1} \ldots b _{j - 1}\}$. Because $a _{k + 1}$ is dependent on $\{a _1 \ldots a _k, b _{k + 1} \ldots b_j\}$, we know there exists some polynomial $f \in \mathbb F[x _1 \ldots x _{j + 1}]$ such that
		      \begin{align*}
			      f(a _1 \ldots a _{k}, b _{k + 1} \ldots b _{j}, X) \neq  0 \land
			      f(a _1 \ldots a _{k}, b _{k + 1} \ldots b _{j}, a _{k + 1})  = 0.
		      \end{align*}

		      We can write $f$ as
		      \begin{align*}
			      f(x _1 \ldots x _{j + 1})
			      = \sum_i f _i(x _1 \ldots x _{j - 1}, x _{j + 1}) x _j ^i.
		      \end{align*}

		      We know that $f(a _1 \ldots a _{k}, b _{k + 1} \ldots b _{j}, X) \neq  0$, therefore at least one of $f _i \neq 0$. Let $g = f _i $ such that $f _i \neq 0$. Because $a _{j + 1}$ is not algebraic over $\{a _1 \ldots a _k, b _{k + 1} \ldots b _{j - 1}\}$, we know that
		      \begin{align*}
			      g(a _1 \ldots a _k, b _{k + 1} \ldots b _{j - 1}, a _{k + 1}) \neq 0.
		      \end{align*}

		      This implies that $f(a _1 \ldots a _k, b _{k + 1} \ldots b _{j - 1}, X, a _{k + 1}) \neq 0$. Since $ f(a _1 \ldots a _{k}, b _{k + 1} \ldots b _{j}, a _{k + 1})  = 0$, we can conclude that $b_j$ is algebraic over $\{a _1 \ldots a _{k + 1}, b _{k + 2} \ldots b _j\}$.

		      Construct $B' = B + a _{k + 1} - b_j$. Having proven that $b_j$ is dependent on a subset of $B'$, it is also dependent on $B'$. All other elements of $B$ are also dependent on $B'$ (because any variable is algebraic over a field extension it generates, i.e. $t$ is algebraic over $\mathbb F(t)$). We recall that $A$ is dependent on $B$, so by the transitivity of algebraic dependence (lemma (\ref{lem:algebraic-dependence-transitivity})), we know that $A$ is algebraically dependent on $B'$. We also notice that $A$ and $B'$ have $k + 1$ elements in common, therefore the statement is proven by the induction hypothesis.
	\end{enumerate}
\end{proof}

We will now show that all bases in our supposed matroid have equal cardinality.

\begin{lemma}\label{lem:algebraic-matroid-equal-size-bases}
	Given some finite subset $E \subseteq \mathbb K / \mathbb F$, maximally independent subsets (bases) of $E$ have equal cardinality.
\end{lemma}

\begin{proof}
	Let $A$ and $B$ be maximal independent subsets of $E$. If both sets are $\emptyset$, then our proof is done. If one set is $\emptyset$ and one isn't, we clearly have a contradiction, as $\emptyset$ is a strict subset of the other set, hence not maximal. We therefore assume the existence of some $a \in A$ and $b \in B$.

	We know that $A$ is maximal, therefore any $b \in B$ is algebraically dependent on $A$. We can therefore conclude that $B$ is algebraically dependent on $B$. We recall that $B$ is also independent, which means we can use lemma (\ref{lem:algebraic-indep-smaller-than-dep}) to prove that $|A| \geq |B|$. We can apply a similar argument to show that $|B| \geq |A|$, which together with our previous statement shows that $|A| = |B|$, proving the lemma.
\end{proof}

\begin{proof}[Proof of theorem (\ref{thm:algebraic-matroids-are-matroids})]
	We will construct the matroid using the basis definition (theorem (\ref{thm:basis-axioms-form-matroid})). We need to first show that at least one exists, and then show that the exchange property holds.
	\begin{enumerate}
		\item[(B1)] We consider the partially ordered set of independent subsets of $E$ ordered by inclusion. We notice that $ \emptyset $ is independent (as the only polynomial of $0$ variables which can be satisfied by $\emptyset $ is the trivial polynomial). Any finite nonempty partially ordered set has at least one maximal element, so a basis exists.
		\item[(B2)] It now remains to prove the exchange property. That is, given two bases $A, B \subseteq \mathcal B$ and some $a \in A \setminus B$, we can find some $b \in B \setminus A$ such that $B - b + a \in \mathcal B$. Let $n = |A| = |B|$ (we know the two bases have equal cardinality from lemma (\ref{lem:algebraic-matroid-equal-size-bases})). Assume the statement is not true. That is, all $b _i \in B$ are dependent on $A - a$, which means $B$ is dependent on $A - a$. It follows from lemma (\ref{lem:algebraic-indep-smaller-than-dep}) that $|B| \leq |A - a|$, which is equivalent to $n \leq n - 1$, hence a contradiction.
	\end{enumerate}
\end{proof}


