\section{Algebraic matroids}

An interesting concept of independence arises in the study of field theory. We will present an interesting class of matroids derived from the concept of algebraic independence. 

TODO: write down the proofs, add example.

\begin{defn}
    A subset $S \subseteq \mathbb K / \mathbb F$ is said to be algebraically independent if it doesn't satisfy any nontrivial polynomial with coefficients in $\mathbb F$.
\end{defn}

Transcendental numbers are special cases of single element algebraically independent sets. Although it is well known that the complex numbers $e$ and $\pi$ are transcendental over $\mathbb{Q} $, it hasn't yet been proven that $\{e, \pi \}$ is algebraically independent.

\begin{theorem}
    Given a finite subset $E \subseteq \mathbb K / \mathbb F$, the pair $(E, \mathcal I)$  where $\mathcal I$ is the set of algebraically independent subsets of $E$ forms a matroid.
\end{theorem}

    Although~\cite{oxley1} offers a proof based on independent sets and the concept of algebraic dependence, we will instead present a much shorter proof based on the basis of our supposed matroid. The part focused on proving the exchange lemma is inspired by~\cite{milne2022}, although the rest will be a lot more simple (no need to involve Zorn's lemma) because we are working in a finite subset.

\begin{proof}
    TODO
\end{proof}
