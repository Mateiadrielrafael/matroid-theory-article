
\newpage
\section{Closure Operator}


We will introduce one more important notion, the one of closure operator. As the name 'operator' suggests, it is a map between objects of the same type. Our objects are, of course, the subsets of the ground set of a matroid. The idea of the closure operator is that it adds to any subset $X \subseteq E$ whatever is left in $E$ that does not change the rank when added to $X$. In doing so it will produce \textit{closed sets} which are in matroid theory called flats, another collection of subsets with distinctive properties. 

Closure also has interesting interpretation in the case of our main examples of representable and graphic matroids. In the case of representable matroids, it adds to $X$ all elements which are in the vector span of $X$ and contained in $E$ (thus it will not increase the vector rank of $E$). In the case of graphic matroids it will add to any set of edges, all edges that create cycles between its sets of vertices (thus will not increase the size of the spanning forest).

\begin{defn}
    Let $M = (E, \mathcal{I})$ be a matroid. The closure operator is a function $\cl: 2^E \to 2^E$ such that

    $$\cl(X) = \{x \in E\; | \; \rank(X \cup x) = \rank(X) \}.$$

    
\end{defn}

The operator has numerous important properties. This time we will immediately prove the reverse statement. Namely, that these characteristic properties \textit{uniquely characterize} the closure operator of a matroid.

\begin{theorem}
Let $M = (E, \mathcal{I})$ be a matroid and $\cl: 2^E \to 2^E$ be a function. Then $\cl$ is the closure operator of $M$ if and only if

\begin{enumerate}
    \item[(CL1)] For any $X \subseteq E$ we have $X \subseteq \cl (X)$
    \item[(CL2)] If $X \subseteq Y \subseteq E$ then $\cl(X) \subseteq \cl(Y)$.
    \item[CL3)] For all $X$ we have $\cl(\cl(X)) = \cl(X)$

    \item[(CL4)] Let $X \subseteq E$ and $b \in 
    \cl(X \cup a) - \cl(X)$ then $a \in \cl(X \cup b)$
\end{enumerate}
\end{theorem}


\begin{proof}
    All the four statements in the \textit{only if} direction (assuming $\cl$ is the closure operator) are pretty straightforward, and we will prove them first. We first assume $\cl$ is the closure operator and would like to show the four properties.

    \begin{enumerate}
        \item[(CL1)] If $X\subseteq E$ is arbitrary, and we pick any element $x \in X$ then $X \cup x = X$. Consequently, $r(X \cup x)= r(X)$ thus $x \in \cl(X)$ by definition of the closure operator which shows the inclusion $X \subseteq \cl(X)$.

        \item[(CL2)] Let $X \subseteq Y \subseteq E$
        and $x \in \cl(X)$. This is equivalent to saying $\rank(X \cup x) = \rank(X)$. Because we would like to show that $x \in \cl(Y)$ we would by definition like that $\rank(Y \cup x) = \rank(Y).$ In particular, we observe that we can without loss of generality assume $x \in \cl(X) - Y$ since the result trivially follows if $x$ is already in $Y$. 
        
        This immediately incentivizes us to check some rank function inequalities containing the sets $X, Y - X$ and these sets containing $x.$ In particular, it would be good if we would come to the inequality 
        $\rank(Y\cup x) \leq \rank(Y)$.
        So our goal is to choose a suitable set $A$ such that $Y \cup A = Y \cup x$ while $\rank(Y \cap A) = \rank(A)$ and we could use the semi-modular inequality of the rank function (R3). More specifically, what we would like is

        $$\rank(\underbrace{Y \cup A}_{ = Y  \cup x}) + \underbrace{\rank(Y \cap A)}_{ = \rank(A)} \leq \rank(Y) + \rank(A).$$

        One $A$ that does exactly this is $A = X \cup x.$ Then we have, since $X \subseteq Y$ that $Y \cup A = Y \cup X \cup x = Y \cup x$, and since we are assuming $x \in \cl(X) - Y$ we have $Y \cap (X \cup x) = X $.

        So $$\rank(Y\cup (X \cup x)) + \rank(X) \leq \rank(Y) + \rank(X \cup x)$$ and since $\rank(X \cup x) = \rank(X)$ by assumption, we have 
        $\rank(Y \cup x)\leq \rank(X)$ what we wanted to show. To conclude, the inequality $\rank(Y) \leq \rank(Y \cup x)$ follows by (R2) so we have $\rank(Y) \leq \rank(Y\cup x) \leq \rank(X)$ implying $\rank(Y \cup x ) = \rank(X)$ so by definition $x \in \cl(Y)$.

        So we have shown the inclusion $\cl(X) \subseteq \cl(Y)$ which was our goal.

        \item[(CL3)] By combining the first and the second property already of closure we have already proven, the inclusion $\cl(X) \subseteq \cl(\cl(X))$  follows. We have to exhibit the reverse one as well, so suppose $z\in \cl(\cl(X))$. This means $\rank(\cl(X) \cup z)= \rank(\cl(X))) = \rank(X)$ where the last inequality follows by the theorem (\ref{rankextension}), namely for all $v \in \cl(X)$ we by definition have $\rank(X \cup v) = \rank(X)$ so by theorem (\ref{rankextension}) we have $\rank(\cl(X)) = \rank(X)$. But now we have, since $X \subseteq \cl(X)$ that $$\rank(X) \leq \rank(X \cup z) \leq \rank(\cl(X)\cup z) = \rank(X)$$

       therefore, the equality holds in all inequalities above. In particular, this means that $\rank(X \cup z) = \rank(X)$ which by definitions means that $z \in \cl(X)$. So the desired inclusion $ \cl(\cl(X))\subseteq \cl(X)$ also holds, which finally implies the desired equality of sets $\cl(\cl(X))= \cl(X).$

        \item[(CL4)] Let $X \subseteq E$ be arbitrary and $b \in \cl(X \cup a) - \cl(X)$. Since, $b \in \cl(X \cup a)$ we have $\rank((X\cup a)\cup b) = \rank((X\cup b) \cup a) = \rank(X \cup a)$ by definition. 

        Now since $b \notin \cl(X)$ then by definition $\rank(X \cup b)\neq \rank(X)$, and because $X \subseteq X \cup b$ we have by (R2) that $\rank(X \cup b) > \rank(X)$. In particular, since the rank function measures the size of the largest independent set inside a given subset, we have that $\rank(X \cup b) = \rank(X)+1.$ 

        Now since $\rank(X \cup a \cup b) = \rank(X \cup a)$, and by the same reason as in the previous paragraph, since $\rank(X \cup a)$ is either $\rank(X)$ or $\rank(X) + 1$, we have it is in fact equal to $\rank(X) + 1$ since $\rank(X \cup a) = \rank(X \cup a \cup b) \geq \rank(X \cup b)= \rank(X) + 1$.

        So we have $r((X\cup b) \cup a) = \rank(X \cup b)$ or in other words, $a \in \cl(X \cup b)$ by definition, which is precisely what we wanted to show.
       
        \end{enumerate}



    Now we prove the complicated reverse direction. We assume we have a finite set $E$ and an arbitrary function $\beta : 2^E \to 2^E$ satisfying the properties of the closure operator listed above.

    Unlike the proofs of the reverse implication of the cryptomorphisms in terms of bases, circuits or rank function, where we had a very straightforward way to spot where the independent sets should appear (for bases they are all of the subsets of members of $\mathcal{B}$, for circuits they are all sets which contain no members of $\mathcal{C}$ and for the rank function they are sets with the property that $\rank(X) = |X|$), this is now not the case, or is at least not immediately apparent.

    So we will not relate the function $\beta$ directly to the independent sets, but rather some other cryptomorphic definition of matroids we have already proven. In particular, observe that if $\cl(X) = E$ holds for some subset $X \subseteq E$ this means that the rank of $X$ is $r(E)$, or in other words, the maximal independent set inside of $X$ is a \textit{basis.}

    So our goal is to look at the collection of all subsets such that $\beta(X)= E$ and prove that the subcollection of its \textit{minimal sets} satisfies the axioms for the collection of bases of a matroid. Therefore, we define $\mathcal{B}' = \{X \subseteq E\; |\; \beta(X) = E\;\}$ and call $\mathcal{B}$ the collection of minimal members of $\mathcal{B}$, for reminder, that are sets which are in $\mathcal{B}$ but do not contain any members of $\mathcal{B}$ as their proper subsets.


\begin{enumerate}

\item[(B1)] We have to prove that $\mathcal{B}$ satisfies the two basis axioms. The first one (B1) is evident since by the first property of closure operator we have the first inclusion in $E\subseteq\beta(E) \subseteq E$ and the second follows because $\beta$ maps into $2^E$. So $E \in \mathcal{B}'$ which in particular implies that $\mathcal{B}'$ and consequently $\mathcal{B}$ are nonempty.

\item[(B2)]
    For the second property (B2) assume $B_1, B_2$ are distinct elements of $\mathcal{B}.$ Because $\mathcal{B}$ consists of \textit{minimal} sets we know that if they are distinct one cannot be a subset of another. So let us pick any $x \in B_1 - B_2$.
    We observe two things, first, $\beta(B_1) = E$ and $\beta(B_1 - x) \neq E$ since $B_1$ is minimal. It is also clear that there exists $y \in B_2 - B_1$ such that $y \in E - \beta(B_1 - x)$, this is because, otherwise all the elements of $B_2$ would be in $\beta(B_1-x)$, but then we would have that $B_2 \subseteq \beta(B_1 - x)$ by (CL2) we would have that $E = \beta(B_2)\subseteq \beta(\beta(B_1 - e)) = \beta(B_1 - e) \neq E$. So by (CL4) we have that we can exchange $x$ and $y$ in the following inclusion.
    
    $$y \in E - \beta(B_1 - x) = \beta((B_1 -x )\cup x)-\beta(B_1 - x)$$ which directly implies that $x \in \beta((B_1 - x)\cup y)$. Now we are done, because this means that 
    
    $$\beta((B_1-x)\cup y) = \beta(\beta((B_1-x )\cup y)) = \beta(\beta((B_1 - x)\cup y)\cup x) \supset \beta(B_1) = E.$$
    
    
    Thus, $(B_1 - x) \cup y$ is in $\mathcal{B}'$ by definition, since its $\beta$ is $E$. We still have to show that $(B_1 - x)\cup y$ is minimal.
    
    Therefore, we will prove separately that all the members of $\mathcal{B}$ have the same size. Let $B_{min}$ be a member of $\mathcal{B}$ with the smallest size and $B_3$ a member of $\mathcal{B}$ such that among all such possible pairs ($B_{min}$ has the smallest number of elements and $B_3$ is another member of $\mathcal{B}$ not equal to $B_{min}$) we have that $|B_{min} \cap B_3|$ is maximal. Since both elements are minimal members of $\mathcal{B}'$ they are not subsets of each other. In particular, this means we can choose $x \in B_{min} - B_3$. 
    
    We know that $\beta(B_{min}-x) \neq E$ and what is more, by the similar argument as before, there has to be $y \in B_3 - \beta(B_{min} - x)$, since otherwise $B_3 \subseteq \beta(B_{min} - x)$ which would mean that $E = \beta(B_3)\subseteq \beta(\beta(B_{min}-x)) = \beta(B_{min} - x)$ which is a contradiction. 
    
    So we have that $y \in \beta((B_{min}-x)\cup x) - \beta(B_{min} -x)$ and (CL4) we thus have $x \in \beta((B_{min}-x)\cup y)$ - we exchange $x$ and $y$. 
    
    However, we see two things. First
    $$\beta((B_{min} - x) \cup y) \supset \beta(B_{min}) = E$$ so $(B_{min} - x) \cup y \in \mathcal{B}'$ and $|B_{min}|  = |(B_{min} - x)\cup y|$ so $(B_{min} - x)\cup y$ is a member of $\mathcal{B}'$ with the smallest possible number of elements. 
    
    Second we see that $|((B_{min}-x)\cup y) \cup B_3| = |B_{min}\cup B_3|+1$ which contradicts the maximality of pair $(B_{min} ,B_3)$. 
    
    So all the elements of $\mathcal{B}$ have the same size, which in  particular implies that $(B_1 - x) \cup y \in \mathcal{B}$ since it is $\beta$  is $E$ what we wanted to show.
    

\end{enumerate}


The last thing to check is if $\beta$ is, in fact, the closure operator for the matroid with the basis set $\mathcal{B}.$ We first introduce some notation, let $M = (E, \mathcal{I})$ be the matroid we are interested in, meaning the one with the basis set $\mathcal{B}$ and we denote its rank function by $\rank'$ and its closure operator by $\cl'$. Our goal is to show that for all subsets $X$ we have that $\cl'(X) = \beta(X)$, i.e. the operators coincide.

We know that $\beta$ and $\cl'$ already do coincide on some sets, precisely the ones that contain a basis, for such $X$ we have $\beta(X) = \cl'(X) = E.$

Second thing to note is that $\beta$ preserves rank, i.e., we always have $\rank'(\beta(X)) = \rank'(X)$ for all $X\subseteq E$. We will prove this by contradiction, let $\rank'(M) = n$. Suppose there is some subset $X$ with $k  = \rank'(X)<\rank'(\beta(X)) = l$ (applying $\beta$ to $X$ has to increase the rank because $X \subseteq \beta(X)$). Then the largest independent set $L$ of $\beta(X)$ has the size $l$. We know $L$ is in some basis $B$ of $\mathcal{B}$, i.e. $L \subseteq B$. Let us observe the set $X \cup (B-L)$.  Its largest independent set has the size at most $k + (n-l)<n$, because the largest independent set of $X$ has size $k$ and we added $|B-L| = n - l$ elements to it, so we increase it by at most $n-l$. The point is that because $k<l$ we have $k + (n-l)<n$, which in particular means that \textit{there is no basis} in $X \cup (B-L)$ and that its rank is less than $n.$


But on the other hand, we have

$$L \subseteq \beta(X) \subseteq \beta(X \cup (B-L))$$ and $$B-L \subseteq \beta(X \cup (B-L))$$ which in particular means that $B \subseteq \beta(X \cup (B-L)$. Finally, this would mean that  
$E = \beta(B)\subseteq \beta(\beta(X \cup (B-L))) = \beta(X \cup (B-L))$ which would imply, by definition of $\mathcal{B}$ that $X \cup (B-L)
\in \mathcal{B}$ - which ultimately means it \textit{contains a basis} of our matroid. But we have already established this is not the case, so we have come upon a contradiction. Therefore, $\beta$ preserves rank.
The fact that $\beta$ preserves rank and looking at the definition of the closure operator, namely that it puts in all the elements which preserve rank, we have thus obtained the inclusion $\beta(X) \subseteq \cl'(X)$ for all subsets $X.$

Finally, suppose for contradiction that for some $X$ the latter inclusion $\beta(X) \subseteq \cl'(X)$ is proper and additionally assume that $X$ is a maximal set with these properties (at some point this will have to stop since for $X =E$ the equality $\beta(E) = \cl'(E) = E$ holds.) First, we pick $x \in E - \cl'(X)$ such that $\cl'(X\cup x) - \cl'(X)$ is nonempty (there exists such an $x$ because $X$ does not contain a basis, otherwise, the equality $\beta(X) = \cl'(X) = E$ would hold).
Because of the maximality of $X$ we hence know that $\beta(X\cup x) = \cl'(X \cup x)$ holds. Also, since the inclusion $\beta(X) \subseteq \cl'(X)$ is proper, we can pick $r \in \cl'(X) - \beta(X)$. In particular, we then have that $r \in \beta(X\cup x) - \beta(X)$ so by (CL4) we have $x \in \beta(X\cup r)$.

However, this means that $X \cup x \subseteq \beta(X\cup r)$ and since $X\cup r \subseteq \cl'(X)$ we have that $\rank'(X\cup r) = 
\rank'(X)$. But on the other hand $$\rank'(X) + 1= \rank'(X\cup x) = \rank'(\beta(X\cup x))\leq \rank'(\beta(\beta(X\cup r))) = \rank'(\beta(X\cup r)) = \rank'(X\cup r) = \rank'(X)$$

which is a contradiction. So our initial assumption that the inclusion $\beta(X) \subseteq \cl'(X)$ is proper for some $X$ was false, which implies that $\beta(X) = \cl'(X)$ for all $X$ and we are done.

\end{proof}


With the definition of closure operator we can define three more important family of subsets of a matroid.

\begin{defn}\label{HP&SS}
    Let $M = (E, \mathcal{I})$ be a matroid with the rank function $\rank$ closure operator $\cl.$ We call a subset $X \subseteq E$ a flat if $\cl(X) = X$ holds. We call a subset $H \subseteq E$ a hyperplane if $H$ is a rank with $\rank(H) = \rank(E)-1$. Finally, we call a subset $S \subseteq E$ a spanning set if $\cl(S) = E$.
\end{defn}

